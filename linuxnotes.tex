% 编译方式: xelatex*2
\documentclass[12pt]{ctexart}
\usepackage{amsmath}
\usepackage{amssymb}
\usepackage{amsthm}
\usepackage{color}
\usepackage{graphicx}
\usepackage{geometry}
\usepackage{hyperref}
\usepackage{marginnote}
\usepackage{mathrsfs}
\usepackage{syntonly}
\usepackage{textcomp}
\usepackage{ulem}
\usepackage{verbatim}
%\syntaxonly
%\geometry{a5paper}
\hyphenation{}
\normalem
\hypersetup{
    colorlinks,
    linkcolor=blue,
    filecolor=pink,
    urlcolor=cyan,
    citecolor=red,
}
\def\b{\boldsymbol}
\def\d{\mathrm{d}}
\def\p{\partial}
\def\ph{\phantom}
\def\t{\text}
\def\ti{\tilde}
\def\v{\vec}
\def\La{\Leftarrow}
\def\Ra{\Rightarrow}
\newcommand{\tabincell}[2]{\begin{tabular}{@{}#1@{}}#2\end{tabular}}
\DeclareMathOperator{\sgn}{sgn}
\DeclareMathOperator{\atanxy}{atan2}
\DeclareMathOperator{\Arg}{Arg}
\theoremstyle{definition}
\newtheorem{definition}{定义}
\newtheorem{theorem}{定理}
\title{Linux Notes}
\author{GasinAn}
\begin{document}

    \maketitle

    \tableofcontents

    \section{配置}

    \begin{table}[htbp]
        \centering
        \begin{tabular}{|c|c|c|c|}
            \hline
            光标向前到首 & 光标向前1词 & 光标向后1词 & 光标向后到尾 \\
            \hline
            Ctrl-a & Alt-b & Alt-f & Ctrl-e\\
            \hline
            向前删除到首 & 向前删除1词 & 向后删除1词 & 向后删除到尾 \\
            \hline
            Ctrl-u & Alt-Backspace & Alt-d & Ctrl-k\\
            \hline
        \end{tabular}
        \caption{热键}
    \end{table}

    \begin{table}[htbp]
        \centering
        \begin{tabular}{|c|c|c|c|c|c|c|c|c|}
            \hline
            intr & quit & eof & eol & swtch & start & stop & susp & discard \\
            \hline
            中断 & 退出 & 文件尾 & 行尾 & 切换shell & 重启输出 & 停止输出 & 挂起 & 丢弃输出 \\
            \hline
            \^{}C & \^{}\textbackslash & \^{}D & <undef> & <undef> & \^{}Q & \^{}S & \^{}Z & \^{}O \\
            \hline
        \end{tabular}
        \caption{热键}
    \end{table}

    \section{打包压缩}

    \begin{table}[htbp]
        \centering
        \begin{tabular}{|c|c|c|}
            \hline
            文件 & 压缩 & 解压缩\\
            \hline
            .tar & tar -cvf & tar -xvf\\
            \hline
            .tar.gz & tar -czvf & tar -xzvf\\
            \hline
        \end{tabular}
        \caption{打包压缩}
    \end{table}

    \section{网络}

    \subsection{SSH}

    \begin{description}
        \item[\texttt{~/.ssh/}] There is no general requirement to keep the entire contents of this directory secret, but the recommended permissions are read/write/execute for the user, and not accessible by others. (700)
        \item[\texttt{~/.ssh/authorized\_keys}] This file is not highly sensitive, but the recommended permissions are read/write for the user, and not accessible by others. It must not be able to be modified or replaced by unauthorized users. (600)
        \item[\texttt{~/.ssh/config}] This file must have strict permissions: read/write for the user, and not writable by others. It may be group-writable provided that the group in question contains only the user. (644)
        \item[\texttt{~/.ssh/id\_rsa}] This file contains sensitive data and should be readable by the user but not accessible by others (read/write/execute). (600)
        \item[\texttt{~/.ssh/id\_rsa.pub}] This file is not sensitive and can (but need not) be readable by anyone. (644)
        \item[\texttt{~/.ssh/known\_hosts}] This file should be writable only by root/the owner and can, but need not be, world-readable. (644)
    \end{description}

\end{document}
